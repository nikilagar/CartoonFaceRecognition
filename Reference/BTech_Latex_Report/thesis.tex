\documentclass{mnnit}

\begin{document}
\title{Your B.Tech. Project Title}
\author{Your Name goes here}
\supervisor{Your Guide Name}

\specialization{Computer Science \& Engineering}
%\specialization{Computer Science \& Engineering}
\beforepreface
\prefacesection{Preface}
A good B.Tech. thesis is one that helps you in furthering your
interest in a specific field of study. Whether you plan to work in
an industry or wish to take up academics as a way of life, your
thesis plays an important role.

Your thesis should judiciously combine theory with practice. It
should result in a realization of reasonably complex system
(software and/or hardware). Given various limitations, it is
always better to extend your predecessor's work. If you plan it
properly, you can really build on the experience of your seniors.

\prefacesection{Acknowledgements}
    Here it will go something like this............It is a great pleasure to thank the giants on whose shoulders I
    stand. First of all, I would like to thank my supervisor ...
\afterpreface

\chapter{Introduction}
This thesis presents the details of writing an B.Tech. thesis
using \LaTeX ~\cite{lamport}. In the previous line, we used the
the \~ \ symbol to leave a small space between the name \LaTeX
~and its citation (appearing in the square brackets). Obviously,
you need to look at the source \TeX ~file to see how this is
actually done in practice.

If you really want to master \LaTeX, you should read the other
excellent book \cite{companion}.

\section{Motivation}
The motivation for this work is...

\subsection{Some Wonderful Minds}
 Donald E. Knuth is the Professor Emeritus of \textbf{The Art of Computer
 Programming} at the \emph{Stanford University}. Leslie Lamport is
 a researcher at \emph{Microsoft corporation}. It is interesting to know
 that Knuth was the creator of \TeX, and Lamport of \LaTeX.

\chapter{The Hypothesis}
In this chapter we shall...

\chapter{Conclusions}
Finally, we give some examples to show how references are created
with bibtex entries. This is a chapter in a book \cite{Kristen}.
This appeared in a conference proceedings \cite{Chen}. However,
this is a bachelor's thesis! \cite{claus}. Finally, this is an
article \cite{Normark}.

Please look at the source \texttt{tex} file for more details. In
fact, this document was created using the same class file that you
are supposed to use while writing the thesis.

Happy \LaTeX ing!!!

\appendix
\chapter{Some Complex Proofs and simple Results}

\bibliographystyle{acm}

\bibliography{references}

\addcontentsline{toc}{chapter}{References}
\end{document}
